% !TeX root = paper.tex


\chapter{結論}\label{conclusion}
本プロジェクトでは,機械学習を用いた電車の車両タイプを判別するシステムの開発を行った.
また,電車が写っている画像や動画をサーバ上で,分類,識別を行い判別結果を出力するWebアプリを作成した.

YouTubeの動画から電車の画像を保存して,トレーニング用とバリデーション用の画像を収集しデータセットを作成した.
そのデータセットとYOLOを用いて,車両タイプの分類モデルと識別モデルの2種類のモデルの作成を行った.
車両タイプを知るためには分類モデルを作成すればよいと考えていたが,分類モデルでは動画に写る電車の車両タイプの判別ができないため,識別モデルを作成した.

%作成したモデルの正解率は車両タイプによって差ができた.JR西日本の在来線の一部の車両タイプについては,短時間で車両タイプを知ることができるようになった.

Webアプリケーションでは,画像の分類,画像の識別,動画の識別の全ての機能での動作が確認された.しかし,一部4G回線を使用して学内のVPNへ接続し動作を検証した際にはアップロードにかなり時間がかかることが確認された.
この学習のデータの収集,モデルの学習,画像・動画に移った特定のものを分類・識別するプロセスはこれからの時代,様々な用途で応用が効くものであると考えられる.

本システムを利用することで電車の知識がない人でも画像または動画に写る一部の車両タイプではその車両タイプが何なのかを知ることができるようになった.
