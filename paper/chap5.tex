% !TeX root = paper.tex


\chapter{結論}
本プロジェクトでは,機械学習を用いた電車の車両タイプを判別するシステムの開発を行った.
また,「たむらたむらたむら」の開発を行った.

YouTubeの動画から電車の画像を保存するという方法を用いてデータセットの作成を試みた.
データセットに含まれる画像の枚数が車両タイプごとに大きな差があったが,画像の枚数が少ない車両タイプの判別結果が悪く,枚数の多い車両タイプの判別結果が良いというわけではなかったため,車体の特徴が鮮明に写っている画像を車両タイプごとに大量に集める必要があったと考えられる.
判別する車両タイプをある路線を走る5種類程度に絞ることでデータ収集とデータの質を精査することをより効率的に行うことができると考えられる.
学習用データの収集を簡略化しようと工夫を重ねたがデータセットの質を確保できなかったため,特徴的な車両の判別は成功して似ている車両の判別にミスが多く見られていると考える.

===========ここから田村==========