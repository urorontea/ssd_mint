% TEX STUDIO MAGIC-COMMAND
% !TeX document-id = {21ffa6e2-6c8f-4532-897c-386dc477f19a}
% !TeX root = presen.tex
% !TeX encoding = utf8
% !TeX TXS-program:compile = lualatex  -synctex=1 -interaction=nonstopmode -halt-on-error %.tex
% !TeX TXS-program:quick = txs:///compile | txs:///view-pdf-internal --embedded
%%%-------------------------------------------------------------------------
%%% PD3プレゼンプレート
%%% 作成: 金沢工大・情報工学科・鷹合研究室
%%%-------------------------------------------------------------------------

\input{tkg_slide.tex}

%%%%%%%%%%%%%%%%%%%%%%%%%%%%%%%%%%%%%%%%%
\renewcommand{\lstlistingname}{リスト}

% 図・表・リストのcaption番号を表示するか/表示しないかを選ぶ
\iffalse
\usepackage[hang,bf,labelformat = empty,labelsep=none,figurename=Y, tablename=X, singlelinecheck=off,justification=centering,labelfont=bf,textfont=bf]{caption} 
\else
\usepackage[hang,bf,labelsep=colon,figurename=図, tablename=表, singlelinecheck=off,justification=centering,labelfont=bf,textfont=bf]{caption} 
\fi

%%%%%%%%%%%%%%%%%%%%%%%%%%%%%%%%%%%%%%%%%
% 
% タイトルスライドのロゴ画像
% フッタ(左)
%%%%%%%%%%%%%%%%%%%%%%%%%%%%%%%%%%%%%%%%%
%  フッタ(左側)

  \MyLogo{\includegraphics[height=1.1cm]{fig/logo/kit_landscape1.pdf}}
% \MyLogo{--- 鷹合研究室 ---} % トップスライドの下部中央

  \lfoot{\includegraphics[height=.75cm]{fig/logo/kit_landscape1.pdf}}
% \lfoot{\small 鷹合研}        % フッタ(左)

%%%%%%%%%%%%%%%%%%%%%%%%%%%%%%%%%%%%%%%%%
% 
% フッタ(中央,右)
%
%%%%%%%%%%%%%%%%%%%%%%%%%%%%%%%%%%%%%%%%%
%\cfoot{\thepage/\pageref{LastPage}} 
\cfoot{\thepage/\pageref{LastPage}}
\rfoot{\small 1EP999} % テーマ番号

%%%%%%%%%%%%%%%%%%%%%%%%%%%%%%%%%%%%%%%%%%%
% ページ番号を1からにしたら,トップスライドの下部のロゴがうまくいかなくなったのでこうしてみた
\fancypagestyle{myfirstpage}
{
  \fancyhf{}
   \fancyfoot[C]{\includegraphics[height=1.1cm]{fig/logo/kit_landscape1.pdf}}
%  \fancyfoot[C]{鷹合研究室}
   \renewcommand{\headrulewidth}{0pt} % removes horizontal header line
}
%%


%%%%%%%%%%%%%%%%%%%%%%%%%%%%%%%%%%%%%%%%%
% 
% ここから下を書き換えて下さい 
%
%%%%%%%%%%%%%%%%%%%%%%%%%%%%%%%%%%%%%%%%%

\title{
{\normalsize 令和5年度 プロジェクトデザインIII}\\\vspace{10mm}
{\LARGE 機械学習を用いた電車の車両タイプの\\判別システムの開発}
}
\date{令和5年9月19日}
\author{
4EP1-68\\ \ruby{野崎}{のざき}\ruby{悠渡}{ゆうと} \and
4EP4-75\\ \ruby{田村}{たむら}\ruby{優祐}{ゆうすけ} 
}



\usepackage{subcaption}
\usepackage{comment}


\begin{document}
\maketitle % タイトルページ
\addtocounter{page}{1}
\thispagestyle{myfirstpage}

%%%%%%%%%%%%%%%%%%%%%%%%%%%%%
\begin{comment}
 \foilhead{\Large 1. はじめに -- 背景と目的 -- \\ 建前編}
\begin{itemize}
 \item 現在,何が問題か(あるいは将来,何が問題になるか)を書く.\\
 世の中には似たようなものがたくさん存在している(動物や車,植物など)
 詳しく知ろうとしたときに,今見ているものが何なのか判別するまでに大きな労力が必要とされている.
 \item その問題に対処するためには,どのようなものがあればよいか(あるいは取り組みが必要)かを書く.\\
 知りたいと思っているモノの写真から,それが何なのか判別できるシステムがあればこれまでよりも簡単に知ることができる.
 \item 本プロジェクトでは何を使ってどんなものを作っているかを書く.\\
 本プロジェクトではYOLOv8を用いて,モノの識別をするシステムの開発を行う
\end{itemize}
\newpage
\end{comment}
\foilhead {\Large 1. 判別対象とする車両タイプ一覧}
電車の画像をいっぱい載せる

\foilhead{\Large 2.データセット作成の流れ}
\begin{enumerate}
	\item YouTubeから1種類の車両タイプのみが写っている動画を保存する
	\item ランダムなフレームを5000枚保存する
	\item ultralytics社が配布するモデル(yolov8n.pt)を利用して2.で保存した画像の識別を行い,電車が一つだけ写っている画像のみ保存する
	\item  (識別用データセット作成時のみ)\\3.で画像の識別をした際に,電車が写っている座標情報をテキストファイルに保存する
\end{enumerate}
各車両タイプごとに行い17種類の車両タイプのデータセットを作成した

\foilhead{\Large 3.動画の保存と連結を行うWebアプリ}
ブラウザの画像を挿入する

デモ動画を流す
\newpage
車両タイプごとの画像の保存枚数の画像を挿入する


\foilhead{\Large 4.モデルの学習}
分類モデルと識別モデルを学習させた.

分類モデルは動画に写る車両タイプを判別できないので,\\
識別モデルを作成した.
\foilhead{\Large 5.作成したモデルの評価}
識別モデルと分類モデルは同様の方法で評価を行う.
\begin{itemize}
	\item 評価方法
	
	モデルを動かしたときに出てくるログから混同行列を作成し,テストデータセットを判別した際の正解数を測定する.
	
	\newpage
	\item 評価
	
	2種類の混同行列を作成すると似たようなものが作成された.
	\vskip\baselineskip %空白行を追加する
	2種類のモデルを動かした結果,外見が似ている車両タイプの正解数は少なく,外見が特徴的なものの正解数は多かった.
	\vskip\baselineskip
	データセットに含まれる各車両タイプの画像の枚数には差があるが、画像の枚数が少ない車両タイプの正解率が低く、画像の枚数が多い車両タイプの正解率が高くなるわけではなかった。
	

\end{itemize}
\newpage
2種類の混同行列の画像を挿入する.

%識別と分類で正解数が低い車両タイプと高い車両タイプを並べて,識別モデルと分類モデルは似たような結果になることを示す.

\foilhead{\Large 5.考察}

データセット内の画像には車体の側面だけが映っているものもあった。

動画や画像の集め方に問題があり、データセットの質が悪くなってしまった.

車両タイプの判別をするときは、基本的に電車の顔が映っている画像を入力するはずなので、データセットに含まれる画像は全て電車の顔が映っているものに限定する必要があると考えられる。

電車の特徴が鮮明に写った画像を集めると判別結果が良くなると考えられる。





\begin{comment}
	

\begin{thebibliography}{99}
\small
\setlength\itemsep{-0.5\zh}%
\bibitem{book1} K.Thompson,D.M.Ritchie,\textbf{"The UNIX Time-Sharing System"},Communications of the ACM, Vol.17, No.7, 1974.
\bibitem{book4} Digital Equipment Corporation: \textbf{PDP11/20-15-r20 Processor Handbook}, 1971.
\bibitem{Preliminary} T.R. Bashkow, \textbf{"Study of UNIX: Preliminary Release of Unix Implementation Document"}, \url{ http://minnie.tuhs.org/Archive/Distributions/Research/Dennis_v1/PreliminaryUnixImplementationDocument_Jun72.pdf}, Jun. 1972.
%\bibitem{book2} K. Thompson,D.M. Ritchie,"UNIX PROGRAMER'S MANUAL",Nov. 1971.
%\bibitem{web0} Warren Toomey, "The Unix Heritage Society", \url{https://www.tuhs.org/}, Dec. 2015.
\bibitem{simh} simh, \textbf{"The Computer History Simulation Project"}, \url{https://github.com/simh/simh}, 参照Mar.14, 2022.
\bibitem{ref0} W.Toomey, \textbf{"First Edition Unix: Its Creation and Restoration"}, IEEE Annals of the History of Computing, 32 (3), pp.74-82, 2010.
%\bibitem{web1} Jim Huang, "Restoration of 1st Edition UNIX from Bell Laboratories", \url{https://github.com/jserv/unix-v1}, 参照Mar.14, 2022.
\bibitem{book3} Diomidis.Spinellis,\textbf{"unix-history-repo"},  \url{https://github.com/dspinellis/unix-history-repo/tree/Research-V1}, 参照Mar.14, 2022.
\bibitem{book5} Digital Equipment Copporation: \textbf{PDP11 Peripherals HandBook}, 1972.
%\bibitem{book6} \url{https://github.com/No000/unix-v1-utils}
%\bibitem{book7} \url{https://github.com/No000/UnixV1-SystemCallTracer}
%\end{thebibliography}
%\end{comment}

%\begin{comment}
\begin{thebibliography}{99}
\small
\setlength\itemsep{-0.5\zh}
\bibitem{book1} ultralytics,  \textbf{"yolov5"}, \url{https://github.com/ultralytics/yolov5},2023.9.18.
\bibitem{book1} ultralytics,  \textbf{"yolov8"}, \url{https://github.com/ultralytics/ultralytics}, 2023.9.18.
\end{thebibliography}
\end{comment}
\end{document} 





